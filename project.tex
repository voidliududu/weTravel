\documentclass[titlepage]{article}
\usepackage{ctex}
\usepackage{graphicx}
\begin{document}
\bibliographystyle{plain}
\title{旅游规划系统}
\author{组员:刘都都}
\date{\today}
\maketitle
\begin{abstract}
\quad \quad 随着人们生活水平的提高,假期旅行成为了一种时尚。但由于人们旅游地信息的了解不足,旅游行为会带有一种盲目性,影响旅游体验。本项目试图从尽可能多的渠道获取旅游周边的信息(地图信息、天气信息、交通流动信息、用户评价信息等),建立用户旅游体验评价体系,开发出旅游行程规划系统和智能推荐系统,并能跟踪用户的具体行程动态地规划旅游安排,以提升用户旅游体验。
\end{abstract}
\tableofcontents
\newpage
\section{问题背景及意义}
\subsection{目前旅游的问题}
近年来,随着节假日旅游也成了人们关注的热点问题,不少人抱怨旅游的体验不佳。总结起来主要有以下问题:

社会问题:
\begin{enumerate}
\item 游客集中向著名景区所在地流动,造成交通压力,影响其他行业运作(如物流业)。
\item 风景名胜区游客过多,增大景区治安难度。
\end{enumerate}

游客问题:
\begin{enumerate}
\item 堵车问题严重,影响出行体验。
\item 景区人员集中,旅游体验不佳。
\item 游客对景区不熟悉,容易走冤枉路,花冤枉钱。
\end{enumerate}
%景点太多如何取舍


这些问题关系着游客的旅行体验。
\subsection{旅游规划软件的现状}
目前有部分企业开发了一些旅游服务类软件,以辅助解决游客的食宿、行程问题,但是它们主要是基于简单的地图服务(寻找游客地理位置周边的服务设施如商店、宾馆等,简单的路线规划),并没有考虑各景点带给旅客的用户体验是不同的,更没有考虑交通、人群的拥挤程度等因素对游客旅行体验的影响。同时,也没有良好的旅游推荐系统,以影响游客的旅游行为。
\subsection{旅游规划系统存在的意义}
一个良好的旅游规划系统能优化用户的旅游过程,给用户提供最佳的旅游体验,同时能在一定程度上缓解交通压力。

在互联网时代,人们早已离不开网络,互联网系统给人们提供着前所未有的便利,但同时也悄悄的影响着用户的行为。特别是近几年,各大互联网公司掌握着用户大量的信息,部分成熟的行为分析系统分析着用户在互联网上的活动,掌握着用户的兴趣、行为偏好等信息,甚至比用户自己都了解他们。大量的推荐系统为用户推荐着所谓用户感兴趣的东西,
他们控制着一切流行的东西。
\section{可行性分析}
\subsection{市场需求}
目前的中国,旅游业是一片欣欣向荣的景象,而游客外出旅游时遇到的问题也是迫切需要解决的。而旅游规划这片市场的涉足者不多。市场需要一个完善的旅游规划系统为游客提供优质的旅游周边信息。
\subsection{社会需求}
旅游规划系统为用户推荐合适的旅游目的地,能够降低旅游的盲目性,从某种程度上来说能起到调度的作用,缓解节假日的交通压力。
\subsection{技术可行性}
随着计算机技术的发展,如今的用户行为偏好分析已经日臻成熟,随着大数据技术的发展,数据的获取、储存与分析已经有了很大的进步,这些技术使得开发这种系统成为了可能。
\section{目标及主要研究内容}
\subsection{项目的预期功能}
\begin{enumerate}
\item 建立一套合理的旅游体验评价系统为用户提供合理的旅游方案
\item 通过分析用户偏好及交通、天气等信息为用户推荐旅游目的地
\end{enumerate}
\subsection{系统的架构设计}
项目现阶段采用B/S架构,整个系统分为4个模块,分别是数据操作模块,第三方服务处理模块,业务核心模块,接口认证与分发模块,各模块的关系如图所示
\includegraphics[scale=0.4]{systemcase.jpg}
以下分别介绍各模块的职责
\subsubsection{数据操作模块}
该模块提供业务所需的数据的存取操作,本项目中我们采用关系型数据库储存业务所需的信息
具体的信息应该包括两大部分,一是用户数据,二是景点数据。

用户数据包括
\begin{enumerate}
\item 用户的基本信息(id,姓名,性别,出生年月,籍贯等)
\item 用户的历史数据,主要包括用户每一次行程的详细记录。精确到用户每一个动作细节(旅游阶段描述,时间)
\item 用户行为偏好的评价指标
\end{enumerate}

景点的信息包括
\begin{enumerate}
\item 景点的基本信息描述
\item 景点的旅游体验评价指标
\end{enumerate}

具体数据的结构化表示将放在第五部分讨论
\subsubsection{第三方服务处理模块}
所谓第三方服务主要指开放的地图服务(如百度地图api)、天气服务、火车票机票查询服务、甚至是社交论坛中关于旅游的话题等。项目需要这些服务提供数据支持。这个模块负责获取第三方服务的数据,为该项目提供使用这些数据的接口。
\subsubsection{业务核心模块}
该模块是项目的核心,应有以下子模块组成
\begin{enumerate}
\item 用户行为分析系统
\item 景点评价系统
\item 智能推荐系统
\item 行程规划系统
\end{enumerate}

系统结构示意图如下右所示
\includegraphics[scale=0.4]{subsystem.jpg}

用户行为分析系统负责将采集到的用户数据量化为用户行为偏好指标,为推荐系统和行程规划系统提供参考;景点评价系统负责从用户数据、用户评价、社交论坛等多渠道评价景点的旅游体验;智能推荐系统从用户和用户群的相似度、景点和景点的相似度,并考虑交通、天气等因素的影响为用户推荐合适的旅游目的地;行程规划系统负责为用户规划合理的旅游计划来优化旅游体验。
\subsubsection{接口认证与分发模块}
该模块负责统一接口的认证服务。该系统完成后可作为一个公共服务供其他第三方个人或企业调用,或者作为一种企业服务提供有偿的调用服务,所以接口的认证与权限的分发机制也属于该项目的一部分。
\subsection{项目具体落实计划}
该项目开源在github上,项目地址为http://github.com/voidliududu/ weTravel。项目比较庞大,凭笔者目前的知识能力水平并不能顺利的完成该项目,所以打算分以下几个阶段逐渐探索:
\begin{enumerate}
\item 假设用户偏好和景点的体验指标已确定的情况下完成行程规划算法的设计
\item 建立景点评价指标
\item 建立用户偏好分析系统
\item 构建智能推荐系统
\item 完善api的认证与分发
\end{enumerate}

本文主要探索第一阶段的工作。

\section{主要研究思路}
\subsection{简单路线规划问题的提出}
本部分主要讨论本项目第一阶段的问题,即在用户偏好和景点的体验指标已确定的情况下完成行程规划算法的设计,首先我们将问题进一步简单化。忽略用户的偏好,知考虑在景点体验指标确定的情况下用户行程路线的最优化设计。问题建立如下:
\begin{quote}
在一个景区中用户每参观一个景点能获得一个确定数值的旅游体验,而从一个景点到另一个景点的过程中同样也会因为旅途劳累、人员拥挤等情况降低一定的旅游体验,给定一个景区,以及一个起点,求一条路径使得用户沿该路径能回到起点并获得最大的旅游体验。
\end{quote}

我们用一个无向图 $<V,E>$来表示这个景区,图的节点记录旅游体验信息$W_v$,图的边记录损失的旅游体验$W_e$,问题转换为求一条回路$L = v_1 v_2 v_3 ... v_1$使下式的值最大$$\sum_{v_i in L} (W_vi - W_ei)$$

我们主要面临的问题有两个。一是这个图可能存在多重边,二是可能存在一些边和节点会被经过多次,我们假定重复经过的边会降低旅游体验,而重复经过的节点不会增加旅游体验。下面我们讨论如何解决这些问题得到最优化解。
\section{关键问题解决的技术路线}
\subsection{路线规划问题的解决思路}
路线规划问题的解决方案有很多,本项目的备选方案主要有遗传算法、蚁群算法。但考虑到遗传算法容易陷入局部最优的情况,本系统将采用蚁群算法来解决路线规划的问题。这种算法具有分布计算、信息正反馈和启发式搜索的特征,在路线规划问题中会有较好的效果。
\subsection{用户行为信息的描述}
该系统需要详细的用户行为信息,所以用户的行为信息的记录是一项非常重要的任务。用户旅游时的行为繁多而复杂,而数据库的记录是结构化的。所以首先要规则化用户的行为信息。由于我们只关注与用户旅游相关的用户行为,而这些行为是有限的,所以我们可以把这些行为归类并具体化。如用户在景点的行为可能有行路、观景、休息等行为。于是我们可以通过时间加行为的方式详细的记录用户的出行信息。

而现在面临的问题是如何合理的划分用户的行为以及如何判断用户的行为。划分用户行为需要进行实地调查,在景区观察统计用户所有可能的行为类型,尽量做到划分合理。而判断用户行为则主要是通过用户携带的手机来判断。通过用户的位置/移动速度、手机陀螺仪、加速度传感器的信息来判断用户的行为,为提高识别的准确率,可采用合适的机器学习算法,用尽可能多的样本训练出识别系统。
\section{测试方法及分析}
测试路线规划问题需要提供以下几类不同的测试用例
\begin{enumerate}
\item 简单图
\item 带多重边的图
\item 最优路径会出现非简单回路的图
\item 部分景点不访问的图
\end{enumerate}

最优路径出现非简单回路的图
\includegraphics[scale=0.4]{graph2.jpg}

部分景点不会被参观的图
\includegraphics[scale=0.4]{graph1.jpg}

对于部分景点不会被访问到的图,说明景点的旅游体验小于到达它话费的代价,这些景点让用户决定是否前往。
\section{结论}
一个良好的旅游规划系统需要各方面的信息支持,需要各领域的技术配合,本项目是对这个系统实现的一次尝试。本文阐述旅游规划系统的组成结构,简要讨论了路径规划算法。因为知识水平原因,本项目有很多尚未讨论的部分,但是日后会依次的完善。
\section*{参考文献}
[1]张宇菲,彭旭,邵光明,陈华友  \quad 旅游路线规划问题 \quad  数学的实践与认识 Vol.46,No. 15 2016.08

[2]孙薇,徐燕锋,孙静怡,杨智伟 \quad 关于旅游路线规划问题的建模与研究 \quad  数学的实践与认识 Vol.46,No. 15 2016.08

[3]朱子江,刘东,刘寿强 \quad 基于用户行为的推荐算法研究 \quad  软件导刊 Vol. 16, No. 18 2017.08
\end{document}
